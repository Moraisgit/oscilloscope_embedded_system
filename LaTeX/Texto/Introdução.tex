% No presente trabalho laboratorial, foi desenvolvido um osciloscópio através de um sistema embebido, com o objetivo de criar uma ferramenta de medição e análise de sinais eletrónicos. Os testes laboratoriais visam verificar a apresentação gráfica do programa, explorar as escalas de medição, obter leituras para diferentes tensões, realizar a calibração do sistema e validar o desempenho do osciloscópio com diferentes formas de onda. 

No presente trabalho de laboratório pretende estudar-se sistemas embebidos, ainda que superficialmente, os quais correspondem a uma combinação de um \textit{hardware} computacional e um \textit{software} projetado para uma função específica. Para tal utilizar-se-á um módulo IoT composto por uma placa micro-controladora com diversas funcionalidades e interfaces e capaz de funcionar como um osciloscópio com um \textit{software} desenvolvido para tal. Com isto, definem-se os objetivos do trabalho: Aprendizagem do circuito de \textit{hardware} base para a realização do trabalho; Desenvolvimento do \textit{software} para implementação do $\mu$Oscilloscope; Teste e ajuste do \textit{software}; Calibração do $\mu$Oscilloscope.

Primeiramente, explica-se o \nameref{sec:Programa desenvolvido} pelo grupo (\texttt{main.py}) com o qual se fizeram \nameref{sec:Testes no simulador} previamente às aulas de laboratório. De seguida apresentam-se os \nameref{sec:Testes no laboratório} onde se esperam ver incongruências na representação dos sinais com a ausência de calibração, mas que após um ajuste experimental, isto é, uma calibração experimental, se espera que essas incongruências sejam erradicadas quase totalmente (sempre sujeito a imperfeições experimentais).